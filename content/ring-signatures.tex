\chapter{Ring Signatures}
\label{chapter:ring-signatures}


Ring signatures are composed of a ring and a signature. Each {\em signature} is generated with a single private key and a set of unrelated public keys. Each {\em ring} is the set of public keys comprising the private key’s public key, and the set of unrelated public keys. Somebody verifying a signature would not be able to tell which ring member corresponds to the private key that created it.
\\

Ring signatures were originally called {\em Group Signatures} because they were thought of as a way to prove a signer belongs to a group, without necessarily identifying him. In the context of Monero they will allow for unforgeable, signer-ambiguous transactions that leave currency flows largely untraceable.
\\

Ring signature schemes display a number of properties that will be useful for producing confidential transactions:

\begin{description}
	
	\item[Signer Ambiguity]
	An observer should be able to determine the signer must be a member of the ring (except with negligible probability), but not which member.\footnote{\label{anonymity_note}Anonymity for an action is usually in terms of an `anonymity set’, which is `all the people who could have possibly taken that action’. The largest anonymity set is `humanity’, and for Monero it is the so-called {\em mixin level} $v$ plus the real signer. Mixin refers to how many fake members each ring signature has. If the mixin is $v$ = 4 then there are 5 possible signers. Expanding anonymity sets makes it progressively harder to find real actors.} Monero uses this to obfuscate the origin of funds in each transaction.
		
	\item[Linkability] 
	If a private key is used to sign two different messages then the messages will become linked.\footnote{\label{linkability_note}The linkability property does not apply to non-signing public keys. That is, a ring member whose public key has been mixed into different signatures will not be linked.} As we will show, this property is used to prevent double-spending attacks in Monero (except with negligible probability).
	 
	\item[Unforgeability]
    No attacker can forge a signature except with negligible probability.\footnote{\label{unforgeability_note}Certain ring signature schemes, including the one in Monero, are strong against adaptive chosen-message and adaptive chosen-public-key attacks. An attacker who can obtain legitimate signatures for messages and corresponding to specific public keys in rings of his choice cannot discover how to forge the signature of even one message. This is called {\em existential unforgeability}; see \cite{MRL-0005} and \cite{Liu2004}.} This is used to prevent theft of Monero funds by those not in possession of the appropriate private keys.
	
\end{description}




\section{Linkable Spontaneous Anonymous Group (LSAG) signatures}
\label{LSAG_section}

Originally (Chaum in \cite{Chaum:1991:GS:1754868.1754897}), group signature schemes required the system be set up, and in some cases managed, by a trusted person in order to prevent illegitimate signatures, and, in a few schemes, adjudicate disputes. Relying on a {\em group secret} is not desirable since it creates a disclosure risk that could undermine anonymity. Moreover, requiring coordination between group members (i.e. for setup and management) is not scalable beyond small groups or within companies.

Liu {\em et al.} presented a more interesting scheme in \cite{Liu2004} building on the work of Rivest {\em et al.} in \cite{rivest-leak-secret}. The authors detailed a group signature algorithm characterized by three properties: {\em anonymity, linkability,} and {\em spontaneity}. In other words, the owner of a private key could produce one anonymous signature by selecting any set of co-signers from a list of candidate public keys, without needing to collaborate with anyone.\footnote{\label{lsag_linkability_note}In the LSAG scheme linkability only applies to signatures using rings with the same members and in the same order, the `exact same ring’. It is really ``one anonymous signature per ring”. Linkable signatures can be attached to different messages.}


\subsection*{Signature}

Ring signature schemes are very Schnorr-like. In fact, our signature scheme in Section \ref{sec:signing-messages} can be considered a one-key ring signature. We get to two keys by, instead of defining $r$ right away, generating a decoy $r'$ and creating a new challenge to define $r$ with.

Let \(\mathfrak{m}\) be the message to sign, \(\mathcal{R} = \{K_1, K_2, ..., K_n\}\) a set of distinct public keys (a group/ring), and \(k_\pi\) the signer's private key corresponding to his public key \(K_\pi \in \mathcal{R}\), where $\pi$ is a secret index. Assume the existence of two hash functions, \(\mathcal{H}_n\) and \(\mathcal{H}_p\),
mapping to integers from 1 to $l$,\footnote{In Monero, the hash function $\mathcal{H}_n(x) = \textrm{sc\textunderscore reduce32}(\mathit{Keccak}(x))$ where $\mathit{Keccak}$ is the basis of SHA3 and sc\textunderscore reduce32() puts the 256 bit result in the range 1 to $l$.} and curve points in EC,\footnote{It doesn’t matter if points from $\mathcal{H}_p$ are compressed or not. They can always be decompressed.}\footnote{Monero uses a hash function that returns curve points directly, rather than computing some integer that is then multiplied by $G$. $\mathcal{H}_p$ would be broken if someone discovered a way to find $n_x$ such that $n_x G = \mathcal{H}_p(x)$.} respectively.

\begin{enumerate}
	\item Compute key image \(\tilde{K} = k_\pi \mathcal{H}_p(\mathcal{R})\).
	
	\item Generate random number \(\alpha \in_R \mathbb{Z}_l\) and fake responses  \(r_i \in_R \mathbb{Z}_l\) for \(i \in \{1, 2, ..., n\}\) but excluding \(i = \pi\).
	
	\item Calculate
	\[c_{\pi+1} = \mathcal{H}_n(\mathcal{R}, \tilde{K}, \mathfrak{m}, [\alpha G], [\alpha \mathcal{H}_p(\mathcal{R})])\]
	
	\item For \(i = \pi+1, \pi+2, ..., n, 1, 2, ..., \pi-1\) calculate, replacing \(n + 1 \rightarrow 1\),\vspace{.2cm}
	\[  c_{i+1} = \mathcal{H}_n(\mathcal{R}, \tilde{K}, \mathfrak{m}, [r_i G + c_i K_i], [r_i \mathcal{H}_p(R) + c_i \tilde{K}])  \] 
	
	
	\item Define the real response $r_\pi$ such that \(\alpha = r_\pi + c_\pi k_\pi \pmod l\).
	
\end{enumerate}

The ring signature contains the signature \(\sigma(\mathfrak{m}) = (c_1, r_1, r_2, ..., r_n) \), the key image $\tilde{K}$, and the ring $\mathcal{R}$.

\subsection*{Verification}

Verification means proving $\sigma(\mathfrak{m})$ is a valid signature created by a private key corresponding to a public key in $\mathcal{R}$, and is done in the following manner:

\begin{enumerate}
    \item Check $l \tilde{K} \stackrel{?}{=} 0$.
	\item  For \(i = 1, 2, ..., n\) iteratively compute, replacing \(n + 1 \rightarrow 1\),\vspace{.2cm}
	\begin{align*}
	c'_{i+1}   = \mathcal{H}_n(\mathcal{R}, \tilde{K}, \mathfrak{m}, [r_i G + c_i {K_i}], [r_i \mathcal{H}_p(\mathcal{R}) + c_i \tilde{K}])
	\end{align*}
	
	\item If \(c'_1 = c_1\) then the signature is valid. Note that $c'_1$ is the last term calculated.
\end{enumerate}

We\marginnote{src/crypto- note\_core/ cryptonote\_ core.cpp check\_tx\_ inputs\_key- images\_do- main()} must check $l \tilde{K} \stackrel{?}{=} 0$ because it is possible to add an EC point from the subgroup of size $h$ (the cofactor) to $\tilde{K}$ and, if all $c_i$ are multiples of $h$ (which we could achieve with automated trial and error using different $\alpha$ and $r_i$ values), make $h$ unlinked valid signatures using the same ring and signing key.\footnote{We are not concerned with points from other subgroups because the output of $\mathcal{H}_n$ is confined to $\mathbb{Z}_l$. For EC order $N = h l$, all divisors of $N$ (and hence, possible subgroups) are either multiples of $l$ (a prime) or divisors of $h$.} This is because an EC point multiplied by its subgroup's order is zero.\footnote{In Monero's early history this was not checked for. Fortunately, it was not exploited before a fix was implemented in April, 2017 (v5 of the protocol) \cite{key-image-bug} .}

To be clear, given some point $K$ in the subgroup of order $l$, some point $K^h$ with order $h$, and an integer $c$ divisible by $h$:
\begin{align*}
    c*(K + K^h) &= cK + cK^h \\
                &= cK + 0
\end{align*}

In\marginnote{src/ringct/ rctOps.cpp} this scheme we store (1+$n$) integers, have one EC key image, use $n$ public keys, and verify with:

\begin{itemize}
    \setlength\itemsep{\listspace}
    \item [\textbf{VBSM}] Variable-base scalar multiplications for some integer $a$, and point $P$: $a P$ \quad [1]
    \item [\textbf{DVBA}] Double-variable-base addition for some integers $a, b$, and point $B$: $a G + b B$ \quad \([n]\)
    \item [\textbf{VBA}] Variable-base additions for some integers $a, b$, and points $A, B$: $a A + b B$ \quad \([n]\)
\end{itemize}



\subsection*{Why it works}

We can informally convince ourselves the algorithm works by going through an example. Consider ring $R = \{K_1, K_2, K_3\}$ with $k_\pi = k_2$. First the signature:
\begin{enumerate}
    \item Create key image: $\tilde{K} = k_\pi \mathcal{H}_p(\mathcal{R})$
    \item Generate random numbers: 	$\alpha$, $r_1$, $r_3$
\begin{align*}
    \intertext{\item Seed the signature loop:}	c_3 &= \mathcal{H}_n(..., [\alpha G], [\alpha \mathcal{H}_p(\mathcal{R})])
    \intertext{\item Iterate: \vspace{-.2cm}} 
        c_1 &= \mathcal{H}_n(..., [r_3 G + c_3 K_3], [r_3 \mathcal{H}_p(\mathcal{R}) + c_3 \tilde{K}])\\
        c_2 &= \mathcal{H}_n(..., [r_1 G + c_1 K_1], [r_1 \mathcal{H}_p(\mathcal{R}) + c_1 \tilde{K}])
\end{align*}
    \item Close the loop by responding: $r_2 = \alpha - c_2 k_2 \pmod{l}$
\end{enumerate}

We can substitute $\alpha$ into $c_3$ to see where the word ‘ring’ comes from:\vspace{.2cm}
\begin{alignat*}{3}
    c_3 &= \mathcal{H}_n(..., [(r_2 + c_2 k_2) G &&] , [(r_2 + c_2 k_2) \mathcal{H}_p(\mathcal{R})&&])\\
    c_3 &= \mathcal{H}_n(..., [r_2 G + c_2 K_2 &&] , [r_2 \mathcal{H}_p(\mathcal{R}) + c_2 \tilde{K}&&])
\end{alignat*} 

Then verification using $\mathcal{R}$, $\tilde{K}$, and $\sigma(\mathfrak{m}) = (c_1, r_1, r_2, r_3)$:
\begin{enumerate}
    \item We use $r_1$ and $c_1$ to compute\vspace{.2cm}
    \begin{align*}
c'_2 &= \mathcal{H}_n(..., [r_1 G + c_1 K_1], [r_1 \mathcal{H}_p(\mathcal{R}) + c_1 \tilde{K}])
    \intertext{\item From when we made the signature, we see $c'_2 = c_2$. With $r_2$ and $c'_2$ we compute\vspace{.2cm}}
c'_3 &= \mathcal{H}_n(..., [r_2 G + c'_2 K_2], [r_2 \mathcal{H}_p(\mathcal{R}) + c'_2 \tilde{K}])
    \intertext{\item We can easily see that $c'_3 = c_3$ by substituting $c_2$ for $c'_2$. Using $r_3$ and $c'_3$ we get\vspace{.2cm}}
c'_1 &= \mathcal{H}_n(..., [r_3 G + c'_3 K_3], [r_3 \mathcal{H}_p(\mathcal{R}) + c'_3 \tilde{K}])
    \end{align*}
\end{enumerate}
\quad No surprises here: $c'_1 = c_1$ if we substitute $c_3$ for $c'_3$.\vspace{-.3cm}



\subsection*{Linkability}

Given a fixed set of public keys \(\mathcal{R}\) and two valid signatures for different messages, \vspace{.1cm}
\begin{align*}
\sigma(\mathfrak{m})   &= (c_1, r_1, ..., r_n)\textrm{ with } \tilde{K}\textrm{, and}\\
\sigma'(\mathfrak{m}')  &= (c_1', r'_1, ..., r'_n)\textrm{ with } \tilde{K}'\textrm{,}
\end{align*}
\quad if \(\tilde{K} =  \tilde{K}'\) then clearly both signatures come from the same signing ring and private key.% because $\tilde{K}= k_{\pi} \mathcal{H}_p(\mathcal{R})$.

While an observer could link $\sigma$ and $\sigma'$, he wouldn’t know which $K_i$ in $\mathcal{R}$ was the culprit without solving the DLP or auditing $\mathcal{R}$ in some way (such as learning all $k_i$ with $i \neq \pi$, or learning $k_\pi$).\footnote{\label{lsag_unforgeable_note}LSAG is unforgeable, meaning no attacker could make a valid ring signature without knowing a private key. If he invents a fake $\tilde{K}$ and seeds his signature computation with $c_{\pi+1}$, then, not knowing $k_\pi$, he can’t calculate a number $r_\pi = \alpha - c_\pi k_\pi$ that would produce $[r_\pi G + c_\pi K_\pi] = \alpha G$. A verifier would reject his signature. Liu {\em et al.} prove forgeries that manage to pass verification are extremely improbable \cite{Liu2004}.}



\section{Back's Linkable Spontaneous Anonymous Group (bLSAG) signatures}
\label{blsag_note}

In the LSAG signature scheme, linkability of signatures using the same private key can only be guaranteed if the ring is constant. This is obvious from the definition \(\tilde{K} = k_\pi \mathcal{H}_p(\mathcal{R})\).

In this section we present an enhanced version of the LSAG algorithm where linkability is independent of the ring’s co-signers.

The modification was unraveled in \cite{MRL-0005} based on a publication by A. Back \cite{AdamBack-ring-efficiency} regarding the CryptoNote \cite{cryptoNoteWhitePaper} ring signature algorithm (previously used in Monero, and now deprecated), which was inspired by Fujisaki and Suzuki's work in \cite{Fujisaki2007}.


\subsection*{Signature}


\begin{enumerate}
	\item Calculate key image \(\tilde{K} = k_\pi \mathcal{H}_p(K_\pi)\).
	
	\item Generate random number \(\alpha \in_R \mathbb{Z}_l\) and random numbers  \(r_i \in_R \mathbb{Z}_l\) for \(i \in \{1, 2, ..., n\}\) but excluding \(i = \pi\).
	
	\item Compute
	\[c_{\pi+1} = \mathcal{H}_n(\mathfrak{m}, [\alpha G], [\alpha \mathcal{H}_p(K_\pi)])\]
	
	\item For \(i = \pi+1, \pi+2, ..., n, 1, 2, ..., \pi-1\) calculate, replacing \(n + 1 \rightarrow 1\),\vspace{.2cm}
	\[  c_{i+1} = \mathcal{H}_n(\mathfrak{m}, [r_i G + c_i K_i], [r_i \mathcal{H}_p(K_i) + c_i \tilde{K}])  \] 
	
	
	\item Define \(r_\pi = \alpha - c_\pi k_\pi \pmod l\).
	
\end{enumerate}

The signature will be \(\sigma(\mathfrak{m}) = (c_1, r_1, ..., r_n) \), with key image $\tilde{K}$ and ring $\mathcal{R}$.

As in the original LSAG scheme, verification takes place by recalculating the value \(c_1\).

Just\marginnote{src/ringct/ rctOps.cpp} like LSAG, in this scheme we store (1+$n$) integers, have one EC key image, use $n$ public keys, and verify with:

\begin{itemize}
    \setlength\itemsep{\listspace}
    \item [\textbf{VBSM}] Variable-base scalar multiplications for some integer $a$, and point $P$: $a P$ \quad [1]
    \item [\textbf{DVBA}] Double-variable-base addition for some integers $a, b$, and point $B$: $a G + b B$ \quad \([n]\)
    \item [\textbf{VBA}] Variable-base additions for some integers $a, b$, and points $A, B$: $a A + b B$ \quad \([n]\)
\end{itemize}

Correctness can also be demonstrated (i.e. `how it works') in a way similar to the LSAG scheme.

The alert reader will no doubt notice the key image $\tilde{K}$ only depends on the true signer's keys. In other words, two signatures will now be linkable if and only if the same private key was used to create them. In bLSAG, rings are just involved in obfuscating each signer’s identity. bLSAG also simplifies the scheme by removing $\mathcal{R}$ and $\tilde{K}$ from the hash that calculates $c_i$.

This approach to linkability will prove to be more useful for Monero than the one offered by the LSAG algorithm, as it will allow detecting double-spending attempts without putting constraints on the ring members used.



\section{Multilayer Linkable Spontaneous Anonymous Group (MLSAG) signatures}
\label{sec:MLSAG}

In order to sign multi-input transactions, one has to sign with \(m\) private keys. In \cite{MRL-0005}, Shen Noether {\em et al.} describe a multi-layered generalization of the bLSAG signature scheme applicable when we have a set of \(n \cdot m\) keys; that is, the set\vspace{.2cm}
\[\mathcal{R} = \{ K_{i,j} \}  \quad \textrm{for} \quad  i \in \{1, 2, ..., n\} \quad \textrm{and} \quad j \in \{1, 2, ..., m\}\] 

where we know the private keys \(\{k_{\pi, j} \} \) corresponding to the subset \(\{K_{\pi, j}\} \) for some index \(i = \pi\).

Such an algorithm would address our multi-input needs if we generalize the notion of linkability.
\begin{description}
	\item[Linkability:] if any private key \(k_{\pi, j}\) is used in 2 different signatures, then these signatures will be automatically linked.
\end{description}


\subsection*{Signature}

\begin{enumerate}
	
	\item Calculate key images \(\tilde{K_j} = k_{\pi, j} \mathcal{H}_p(K_{\pi, j})\) for all \(j \in \{1, 2, ..., m\}\).
	
	\item Generate random numbers  \(\alpha_j \in_R \mathbb{Z}_l\), and \(r_{i, j} \in_R \mathbb{Z}_l\) for \(i \in \{1, 2, ..., n\}\) (except \(i = \pi\)) and \(j \in \{1, 2, ..., m\}\).
	
	\item Compute\footnote{Monero\marginnote{src/ringct/ rctSigs.cpp MLSAG\_gen()} MLSAG uses so-called `key prefixing', which, when implemented, was thought to allow better security proofs for multi-user Schnorr-like signatures. Recent research suggests it isn't necessary. \cite{key-prefix-paper} Each challenge contains an explicit public key like this:
	\[
	c_{\pi+1} = \mathcal{H}_n(\mathfrak{m}, K_{\pi, 1}, [\alpha_1 G], [\alpha_1 \mathcal{H}_p(K_{\pi, 1})], ..., K_{\pi, m}, [\alpha_m G], [\alpha_m \mathcal{H}_p(K_{\pi, m})])
	\]} 
	\[
	c_{\pi+1} = \mathcal{H}_n(\mathfrak{m}, [\alpha_1 G], [\alpha_1 \mathcal{H}_p(K_{\pi, 1})], ..., [\alpha_m G], [\alpha_m \mathcal{H}_p(K_{\pi, m})])
	\]
	
	\item For \(i = \pi+1, \pi+2, ..., n, 1, 2, ..., \pi-1\) calculate, replacing \(n + 1 \rightarrow 1\),\vspace{.2cm}
	\[  c_{i+1} = \mathcal{H}_n(\mathfrak{m}, [r_{i, 1} G + c_i K_{i, 1}], [r_{i, 1} \mathcal{H}_p(K_{i, 1}) + c_i \tilde{K}_1], 
	..., [r_{i, m} G + c_i K_{i, m}], [r_{i, m} \mathcal{H}_p(K_{i, m}) + c_i \tilde{K}_m])  \] 
	
	
	\item Define \(r_{\pi, j} = \alpha_j - c_\pi k_{\pi, j} \pmod l\).
	
\end{enumerate}

The signature will be \(\sigma(\mathfrak{m}) = (c_1, r_{1, 1}, ..., r_{1, m}, ..., r_{n, 1}, ..., r_{n, m}) \), with key images $(\tilde{K}_1, ...,  \tilde{K}_m)$.

%One way to think about MLSAG is that there are $m$ sub-loops of size $n$, and in each sub-loop we know a private key at index $i = \pi$ ($m \cdot n$ total public keys). The signature algorithm ties together a ‘stack’ of keys at each stage $c$, composed of one key from each sub-loop. bLSAG is the special case where $m = 1$.

\subsection*{Verification}

The verification of a signature is done in the following manner:

\begin{enumerate}
    \item For all $j \in \{1,...,m\}$ check $l \tilde{K}_j \stackrel{?}{=} 0$.
	\item  For \(i = 1, ..., n\) compute, replacing \(n + 1 \rightarrow 1\),\vspace{.2cm}
	\begin{align*}
	c'_{i+1} = \mathcal{H}_n(\mathfrak{m}, [r_{i, 1} G + c_i K_{i, 1}], [r_{i, 1} \mathcal{H}_p(K_{i, 1}) + c_i \tilde{K}_1], 
	..., [r_{i, m} G + c_i K_{i, m}], [r_{i, m} \mathcal{H}_p(K_{i, m}) + c_i \tilde{K}_m]) 
	\end{align*}
	
	\item If \(c'_1 = c_1\) then the signature is valid.
\end{enumerate}


\subsection*{Why it works}

Just as with the original LSAG algorithm, we can readily observe that

\begin{itemize}

\item If \(i \ne \pi \), then clearly the values \(c'_{i + 1}\) are calculated as described in the signature algorithm.

\item If \(i = \pi\) then, since \(r_{\pi, j} = \alpha_j - c_\pi k_{\pi, j} \) closes the loop,\\
\begin{alignat*}{6}  
  r_{\pi, j} G + c_\pi K_{\pi,j} &= (\alpha_j - c_\pi k_{\pi, j}) G + c_\pi K_{\pi,j} = \alpha_j G \\  
    \intertext{and}
  r_{\pi, j} \mathcal{H}_p(K_{\pi, j}) + c_\pi \tilde{K}_j &= (\alpha_j - c_\pi k_{\pi, j}) \mathcal{H}_p(K_{\pi, j}) + c_\pi \tilde{K}_j = \alpha_j \mathcal{H}_p(K_{\pi, j})\\ 
\end{alignat*}
   In other words, it holds also that \(c'_{\pi + 1} = c_{\pi+1}\).

\end{itemize}


\subsection*{Linkability}

If a private key \(k_{\pi, j}\) is re-used to make any signature, the corresponding key image \(\tilde{K}_j\) supplied in the signature will reveal it. This observation matches our generalized definition of linkability.\footnote{As with LSAG, linked MLSAG signatures do not indicate which public key was used to sign it. However, if the key image's sub-loops' rings have only one key in common, the culprit is obvious.}


\subsection*{Space and verification requirements}

In\marginnote{src/ringct/ rctOps.cpp} this scheme we store (1+$m*n$) integers, have $m$ EC key images, use $m*n$ public keys, and verify with:

\begin{itemize}
    \setlength\itemsep{\listspace}
    \item [\textbf{VBSM}] Variable-base scalar multiplications for some integer $a$, and point $P$: $a P$ \quad [m]
    \item [\textbf{DVBA}] Double-variable-base addition for some integers $a, b$, and point $B$: $a G + b B$ \quad \([m*n]\)
    \item [\textbf{VBA}] Variable-base additions for some integers $a, b$, and points $A, B$: $a A + b B$ \quad \([m*n]\)
\end{itemize}
